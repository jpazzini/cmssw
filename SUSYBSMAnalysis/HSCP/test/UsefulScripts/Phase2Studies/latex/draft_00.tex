\documentclass[11pt,twoside,a4paper]{article}

\usepackage{rotating,booktabs,graphicx}

\title{HSCP for Phase2 Trk TDR}

\begin{document}

\section{Introduction}

Several theoretical extentions of the Standard Model (SM), including Super Symmetry (split SUSY and MSSM), Gauge mediated Symmetry Breaking (GMSB), Hidden-Valley, Extra dimensions, and several others, predict the existance of heavy particles with long lifetime.
If produced with lifetime larger that a few nanoseconds, these heavy stable (or quasi-stable) charged particles (HSCPs), can pass through part (or the totality) of the detector before decaying, and therefore appear stable like pions or kaons.
Depending on the mass and the charge of the HSCP particle, anomalously high rates of energy losses through ionization ($\mathrm{d}E/\mathrm{d}x$) with respect to the typical energy loss for SM particles ($\approx 3 \mathrm{MeV}/\mathrm{cm}$ for particles with 10-1000 GeV momentum) can be expected. 
This feature allows for a sensitivity to massive particles with lower velocity ($\beta = v/c$) compared to SM particles, or to particles with charge $\mathrm{Q} > 1|e|$.

\section{Samples and Event Selection}

Two benchkarm scenarios are employed to account for a wider range of signatures of HSCP production at LHC.

A first signal is pair production of gluino ($\tilde{g}$) that interact via the strong interaction and hadronize with SM quarks to form R-hadrons. 
Signal gluino have been generated with masses in the range 600-2600 GeV under the Split SUSY scenarios.
Gluinos are generated under the high squark mass (10 TeV) assumption using PYTHIA v8.xxx [35] with the default tune CUETP8M1.
The fraction, $f$ , of produced $\tilde{g}$ hadronizing into a $\tilde{g}$-g state (R-gluonball) is an unknown parameter of the hadronization model and affects the fraction of R-hadrons that are neutral at production. 
For this study results are obtained for $f = 0.1$.

The second kind of signal consists of lepton-like HSCPs. 
The benchmark chosen for lepton-like HSCP is the minimal gauge mediated supersymmetry breaking (mGMSB) model, where production of supersymmetric quasi-stable leptons ($\tilde{\tau}$) can proceed either directly or via production of heavier supersymmetric particles (mainly squarks and gluino
pairs) further decaying to one or more $\tilde{\tau}$ particles. 
The mGMSB model is chosed as benchmark using the SPS7 slope, which has the stau as the next-to-lightest supersymmetric particle (NLSP). 
The particle mass spectrum and the decay table are produced with the program ISASUGRA version 7.69. The mGMSB parameter $\Lambda$ is varied from 31 to
160 TeV, with fixed parameters $\mathrm{N}_{\mathrm{mes}} = 3$, $\tan \beta = 10$, $\mu > 0$, $\mathrm{C}_{\mathrm{grav}} = 10000$, and $\mathrm{M}_{\mathrm{mes}}/\Lambda = 2$. 
The large value of $\mathrm{C}_{\mathrm{grav}}$ results in a long-lived stau, while $\Lambda = 31-510$ TeV gives a stau mass of 200 to
1600 GeV. 
Signal is generated using PYTHIA v8.xxx with the default tune CUETP8M1.

Backgroud samples have been used to simulate tracks produced as the result of proton-proton interactions in the CMS detector.
For this purpose, Minimum Bias samples have been used, generated 
To simulate the effect of high transverse momentum muons traversing the detector, a samples of Drell-Yan production is also used.


All the signal and backgroud samples used in this analysis have been subjected to the full simulation of the CMS detector with the 2023-D4 geometry. 

\begin{sidewaystable}
\small
\centering
\caption{Simulated samples used for this study. All the samples have been produced in three different pile-up configurations: NoPU, 140PU, and 200PU.\label{tab:samples}}
\begin{tabular}{ll}
\toprule
Sample & $\sigma$ \\
\midrule
\verb|/HSCPgluino_M_600_TuneCUETP8M1_14TeV_pythia8/PhaseIIFall16DR82-*_90X_upgrade2023_realistic_v1-v1/GEN-SIM-RECO|	& xxx \\
\verb|/HSCPgluino_M_1000_TuneCUETP8M1_14TeV_pythia8/PhaseIIFall16DR82-*_90X_upgrade2023_realistic_v1-v1/GEN-SIM-RECO|	& xxx \\
\verb|/HSCPgluino_M_1800_TuneCUETP8M1_14TeV_pythia8/PhaseIIFall16DR82-*_90X_upgrade2023_realistic_v1-v1/GEN-SIM-RECO|	& xxx \\
\verb|/HSCPgluino_M_1400_TuneCUETP8M1_14TeV_pythia8/PhaseIIFall16DR82-*_90X_upgrade2023_realistic_v1-v1/GEN-SIM-RECO|	& xxx \\
\verb|/HSCPgluino_M_2600_TuneCUETP8M1_14TeV_pythia8/PhaseIIFall16DR82-*_90X_upgrade2023_realistic_v1-v2/GEN-SIM-RECO|	& xxx \\
\verb|/HSCPgluino_M_2200_TuneCUETP8M1_14TeV_pythia8/PhaseIIFall16DR82-*_90X_upgrade2023_realistic_v1-v1/GEN-SIM-RECO|	& xxx \\
\midrule
\verb|/HSCPppstau_M_200_TuneCUETP8M1_14TeV_pythia8/PhaseIIFall16DR82-*_90X_upgrade2023_realistic_v1-v1/GEN-SIM-RECO|	& xxx \\
\verb|/HSCPppstau_M_432_TuneCUETP8M1_14TeV_pythia8/PhaseIIFall16DR82-*_90X_upgrade2023_realistic_v1-v1/GEN-SIM-RECO|	& xxx \\
\verb|/HSCPppstau_M_651_TuneCUETP8M1_14TeV_pythia8/PhaseIIFall16DR82-*_90X_upgrade2023_realistic_v1-v1/GEN-SIM-RECO|	& xxx \\
\verb|/HSCPppstau_M_871_TuneCUETP8M1_14TeV_pythia8/PhaseIIFall16DR82-*_90X_upgrade2023_realistic_v1-v1/GEN-SIM-RECO| 	& xxx \\
\verb|/HSCPppstau_M_1218_TuneCUETP8M1_14TeV_pythia8/PhaseIIFall16DR82-*_90X_upgrade2023_realistic_v1-v1/GEN-SIM-RECO|	& xxx \\
\verb|/HSCPppstau_M_1599_TuneCUETP8M1_14TeV_pythia8/PhaseIIFall16DR82-*_90X_upgrade2023_realistic_v1-v1/GEN-SIM-RECO|	& xxx \\
\midrule
\verb|/MinBias_*_TuneCUETP8M1_14TeV-pythia8/PhaseIIFall16DR82-*_90X_upgrade2023_realistic_v1-v1/GEN-SIM-RECO|		& xxx \\
\midrule
\verb|/DYJetsToLL_M-50_TuneCUETP8M1_14TeV-madgraphMLM-pythia8_ext1/PhaseIIFall16DR82-*_90X_upgrade2023_realistic_v1-v1/GEN-SIM-RECO|	& xxx \\
\bottomrule
\end{tabular}
\end{sidewaystable}

Signal events are selected requiring the reconstruction of a muon with transverse momentum $p_{\mathrm{T}} > 55$ GeV, as reconstructed form the particle-flow algorithm.

All events are further required to have a candidate track with $p_{\mathrm{T}} > 55$ GeV (as measured in the tracker detector only), with a relative uncertainty on $p_{\mathrm{T}}$ less than 25\%, pseudorapidity in the tracker acceptance $|\eta| < 4$, track fit $\chi^2/\mathrm{dof} < 5$, and magnitudes of the longitudinal and transverse impact parameters with respect to the primary vertex less than 0.5 cm. 
Candidates tracks must have at least XXX measurements in the silicon pixel detector.

\section{Ionizing Particle Discriminator}

An estimator of the degree of compatibility of the track with the hypothesis of it being generated by a MIP is defined.
The $\mathrm{d}E/\mathrm{d}x$ discriminator $I_{\mathrm{as}}$ is define as:

$$
I_{\mathrm{as}} = \frac{3}{N} \times \left( \frac{1}{12N} + \sum_{i=1}^{N} \left[ P_i \times \left( P_i - \frac{2i-1}{2N} \right)^2 \right] \right)
$$

where $N$ is the number of measurements in the silicon-tracker detectors, $P_i$ is the probability for a minimum-ionizing particle to produce a charge smaller or equal to that of the $i$-th measurement for the observed path length in the detector, and the sum is over the track measurements ordered in terms of increasing $P_i$.

\begin{figure}
\centering
\includegraphics[width=.5\textwidth]{figures/dummy.pdf}
\caption{DeDxVsP\label{fig:dedxvsp}}
\end{figure}

The HSCP candidates are selected using the $I_{\mathrm{as}}$ discriminator by requiring tracks with  criteria are $p_{\mathrm{T}} > 65$ GeV and $I_{\mathrm{as}} > 0.3$. 
The backgound yield is estimated from the MinBias and DYJets samples with a "ABCD" procedure by correcting the number of events passing the $I_{\mathrm{as}}$ requirement only, for the ratio of the events yields passing the $p_{\mathrm{T}}$ requirement only divided by the number of events failing both requirements.

\begin{figure}
\centering
\includegraphics[width=.5\textwidth]{figures/dummy.pdf}
\caption{ROC\label{fig:roc}}
\end{figure}

\section{Expected Limits}

The expected perfomances of the analyis in absence of signal are shown in terms of cross-section reach, defined as the cross-section for which we expect to observe signal with a significance of at least 5 standard deviations (5$\sigma$).
The results are presented for the three pileup scenarios (NoPU, 140PU, 200PU) for an integrated luminosity of 3000 $\mathrm{fb}^{-1}$.

\begin{figure}
\centering
\includegraphics[width=.5\textwidth]{figures/dummy.pdf}
\caption{Limit\label{fig:limit}}
\end{figure}

\end{document}

